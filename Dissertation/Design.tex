\chapter{Design}
\section{Overview}
The design for this project could be separated into two major parts - the design of the game itself and the design of the computer player. It is important to note that the design process ran essentially concurrently with the implementation of the program, as I followed a more iterative approach, rather than planning everything before starting work on the program. As well as that, the design and implementation of the game client came entirely before the design and implementation of the computer player - meaning, I had only a vague idea how the computer player is going to work or what approach it is going to take to making decisions before I finished with the game client.

The goals of my game client design were, therefore, twofold. The primary goal was still to create a working game client that a user could interact with and play the game through, and that would have all the features and rules that the game requires. The secondary goal, however, was to make my implementation extendible, so that different implementations of a computer player could be added on top of it, with little, if any, rewriting of the game client code. While I succeeded at the primary goal, and the client is fully functional, I mostly failed at the secondary goal - when implementing the computer players, multiple rewrites of the game client code were necessary, and these rewrites were one of the most time-consuming parts of this entire project.

\section{Game Client Design}
\subsection{Game Engine}
I chose to use Unity as the game engine for this project. This choice was mainly motivated by the fact that this was the game engine I was by far most comfortable with, and had used it for multiple previous hobby game projects. However, a strong secondary reason for my choice was the fact that \textit{Risk: Global Domination}, an officially licensed, commercially available implementation of the game, was also made using Unity as the game engine. This showed me that, not only is an implementation of Risk practical and viable in Unity, but also that the industry professionals responsible for making a commercial version of the game were evidently of the opinion that Unity is the best engine to use for their version of the game. While this did not guarantee that it was the best choice for my implementation, combined with my knowledge of the engine, this led me to choosing Unity.

Other game engines that were considered, such as libGDX and Godot. Their main draw was the fact that this was going to be a light-weight, rudimentary implementation of the game, and that it did not perhaps need a large, fully-featured 3D game engine such as Unity. However, the lack of a GUI when developing with libGDX was too major of a downside for me to get past. Godot, meanwhile, was only slighly lighter than Unity, and would have required me to code the game in GDScript, a language made for the Godot engine. I was wary of having to learn a brand new programming language on top of a new engine and all the other work required for this project, and therefore decided not to go with Godot.
