\graphicspath{ {./Images/} }
\chapter{Background}
\section{Rules of Risk}
\label{rulesOfRisk}
Risk \cite{riskrules} is a 2-6 player strategy board game, in which players try to conquer the world. The game is played on a game board representing the entire world, which is divided into 42 territories, with each of those territories belonging to one of 6 continents. Players start by controlling an equal amount of territories each, and must earn more troops and attack territories of other players in order to win.

A game starts by each player placing one troop on a territory, thereby claiming it. Once every territory is claimed, players continue placing troops on their already claimed territories, until each player has placed a predetermined number of troops, which can range from 40 to 20, depending on how many players there are. After this set-up phase is complete, players take their turns in sequence. A turn begins with the player distributing newly earned troops on their territories. The amount of new troops they earn is predetermined based on territories and continents - the player gets 1 troop for every 3 territories they control, rounded down, with a minimum of 3 troops, even if they control less than 9 territories. Furthermore, the player gets extra troops if they control every territory in a continent, with the exact number they get depending on the specific continent - ranging from 2 troops for controlling Australia up to 7 troops for controlling Asia.

After their new troops are deployed, the player can start making attacks. Territories can be attacked from adjacent territories that contain more than 1 troop. It is never allowed, under any circumstances, to leave a territory with 0 troops, therefore, for example, when trying to attack from a territory with 8 troops, only 7 troops can participate in the attack - there must always be at least 1 troop that is left in a territory. The attack itself is decided via dice roll. The attacker rolls 3 dice or less, if they have less than 3 attacking troops. The defended rolls 2 dice or less, if they have less than 2 defending troops. After the dice are rolled, the highest rolled numbers from both players are matched, and the second highest numbers from both players are matched. For each match, whoever has the higher roll makes the other lose 1 troop. Ties are won by the defender. For example, say an attacker rolled dice with values \texttt{5, 3, 2}, and the defender rolled \texttt{5, 1}. The highest rolls are a 5 for both players, so the 5 are matched against each other. The second highest rolls are 3 for the attacker, 1 for the defender. The first match is a tie, therefore, the defender wins, and the attacker loses 1 troop. In the second match, the attacker rolled higher, so the defender loses 1 troop. Final result: both players lose 1 troop. The attacker can choose to stop attacking after each dice roll is resolved, or can keep rolling dice until either player has no troops left to fight for that territory.

If an attack is successful - that is, if the defender loses all of their defending troops - the attacking player can move some of the troops they were attacking with into the newly captured territory. The minimum number of troops they have to move is the number of dice they rolled on their last attack (most likely, 3). The maximum number is the total number of attacking troops (keeping in mind that, once again, at least 1 troop must be left behind). The player can make any number of different attacks in their turn, until either there are no more legal attacks to make, or they decide to stop attacking. After a player decides to stop attacking, they can make one reinforcing move between 2 of their own (adjacent) territories, moving any number of troops between those two territories. Once this move is done, the turn ends, and the next player starts their turn - deploying new troops, attacking, then reinforcing their own territories.

If a player has captured at least 1 territory in their turn, at the end of said turn, they receive a card. There are 44 cards in total - 42 representing each territory on the board, and 2 "wild cards". Each non-wild card also has a picture on it - of either infantry, cavalry, or artillery. When deploying new troops at the start of their turn, a player may trade-in their cards for extra troops. They can trade in a set of 3 cards, which must have either matching pictures or 3 different pictures. A wild card may be used in place of any picture type. If a player has 5 or more cards at the start of their turn, they must trade a set in (as it is guaranteed that they have a valid set). The first set traded in gives 4 troops, and every set traded in by any player increases the reward of the next set to be traded in. Furthermore, if the player owns any of the territories represented by the cards that are being traded in, they receive 2 extra troops from that set. Finally, if a player is defeated - meaning they do not control any territories - their hand of cards is given to the player that captured their final territory.

\section{Monte Carlo Tree Search Theory}
\label{MCTSTheory}

Monte Carlo tree search (MCTS) \cite{Coulom2007MonteCarlo} is a search algorithm